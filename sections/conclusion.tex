%Conclusion: A summary of the main findings and their implications, along with suggestions for future research or practical applications.
%Summarize the key points of your research statement.
%Reinforce the importance of your research and the potential contributions it can make to the field.

\section{Conclusion}

Fallout is a game you can replay many times to change how the game ends.

You need to try many times to find a character which works in the game world.

You find you need to make many saves, partially because of crash bugs, but also because one can easily make regretful and unsatisfying decisions.

There are design flaws, such as objects blending in with the environment and clunky user interfaces where interactivity points (signifiers) are unclear.

Fallout is authored, it has no procedurally generated world. This might limit the enjoyment of the player as it limits re-playability.

During my research I could not help but think how lucky human beings currently are. As the late mathematician Paul Erdös said, God must have created us to enjoy our suffering \autocite[4]{hoffman_man_1998}. Art tends to be overly soft. Children outgrow many of the games they play as they become older, as they become better equipped to handle greater difficulties \autocite[99]{hiwiller_players_2016}. The choices and experiences in Fallout are dark. Our modern lives are light in comparison.

Fallout is built around a system. As you make choices, the system reacts and continues to react. Your enemies grow stronger. Your factions will agree to help. The results of your choices and the effects they have on the system are felt by the player from start to finish.
