%Methods: A description of the research methods and procedures used to conduct the study, including information on data collection, participants (if applicable), and any statistical or analytical techniques employed.

%\section{Methods}


% A game is
% A role-playing game is

\section{Player choices}

In Fallout, you play a character from Vault 13, a survivor of the post-apocalypse. Your character has been chosen to leave the vault, and its community, to secure a new water chip from the outer world. As the vault's critical water chip has malfunctioned, you only have 150 days to find a new chip and save your vault.

\subsection{Character selection and creation}

The first thing you do in Fallout is choose a character to play.

All the characters you can choose are balanced. The characters you choose from have different strengths and weaknesses.  If you design your own character, you get to set the character's name, age, traits, skills and attributes.

Every character in the game, enemy or foe, is controlled by the same SPECIAL role-playing system.

%After customizing their character, the player must scour the surrounding wasteland for a computer chip that can fix the Vault's failed water supply system. They interact with other survivors, some of whom give them missions, and engage in turn-based combat where they battle until their action points are depleted.

%Character creation
%Fallout is a role-playing video game. The player begins by selecting one of three characters, or one with player-customized attributes.[2] The protagonist, known as the Vault Dweller,[b] has seven primary statistics that the player can set: strength, perception, endurance, charisma, intelligence, agility, and luck.[6] Each statistic may range from one to ten, provided their sum does not exceed 40.[7] Two other statistics set during character creation are skills and traits.[8] All 18 skills are learned abilities, their effectiveness determined by a percentage value. Their initial effectivenesses are determined by the primary statistics, but three can be given a 20% boost.[9] Traits are character qualities with both a positive and negative effect; the player can pick two from a list of sixteen.[10][11] During gameplay, the player can gather experience points through various actions. For gathering experience points, the player will level up and may increase their skills by a set number of points.[8] Every three levels, the player can grant themself a special ability, or perk.[12] There are 50 perks and each has prerequisites that must be met. For example, "Animal Friend", which prevents animals from attacking the player character, requires the player to be level nine, have an intelligence of five, and have an outdoorsman skill of 25%.[13]

%Exploration and combat
%There are three boxes in the middle part of the screenshot. The top box contains Killian Darkwater's head and shoulders. The middle box contains dialogue from Killian, who is thanking the player for saving him from his assassination and asking them to find evidence implicating Gizmo in requesting the assassination. The bottom box contains possible dialogue choices for the player.
%Dialogue with Killian Darkwater, an example of a non-player character with a talking head. He is offering the player a quest to complete.
%In Fallout, the player explores the game world from a trimetric perspective and interacts with non-player characters (NPCs). Characters vary in their amount of dialogue; some say short messages, while others speak at length. Significant characters are illustrated with 3D models, known as "talking heads", during conversations.[10][14] The player can barter with other characters or buy goods using bottle caps as currency.[15] The game has companions that the player can recruit for exploration and combat, although they cannot be directly controlled.[10]

%There are three main quests where completion is required, two of them given after completion of the first one. The first main quest has a time limit of 150 in-game days; the game ends if the player fails to complete it within the allotted time.[16][17][c] Some characters give the player side quests; if the player solves them, they receive experience points.[8][16] The player can use the PIPBoy 2000, a portable wearable computer that tracks these quests.[19] Many quests feature multiple solutions; they can often be completed through diplomacy, combat, or stealth, and some allow solutions that are unconventional or contrary to the original task.[20] Based on how they completed quests, the player can earn or lose karma points, which determine how others treat them.[10] The player's actions dictate what future story or gameplay opportunities are available and the ending.[16][21][8]

%Combat is turn based and uses an action-point system. During each turn, multiple actions may be performed by the player until they run out of action points.[22] Different actions consume different amounts of points.[23] The player can rapidly switch between two equipped weapons,[24] and may acquire a diverse range of weapons,[16] many of which can target specific areas of enemies.[25] Melee (hand-to-hand) weapons typically have two attacks: swing or thrust. If the player has equipped no weapon, they can punch or kick.[26]

\subsubsection{Attributes}

In the SPECIAL system, there are seven main attributes: strength, perception, endurance, charisma, intelligence, agility, and luck. Each will be set in a range from 1 to 10. 

\subsubsection{Traits}

Traits modify aspects of gameplay, including the player's attributes and skills. Traits are optional, each trait has a benefit and a penalty, and can only be chosen at the very beginning of the game. An example of a trait is ``Gifted'', which makes the player gain an extra point to each attribute but has the penalty of rewarding the player fewer skill points at level up, making the player character more fixed and less dynamic as the game goes on.

\subsubsection{Skills}

Skills are the learnt abillities of a character. Skills can be improved by reading specific books or going up in level. There are combat skills which dictate a character's proficiency with various weapons, such as small guns, or energy weapons. There are also skills which are active and passive, which affect how well a character can haggle down prices on wares, how easily a character can sneak undetected and repair equipment. The player can choose where to apply new skill points at each level up.

\subsubsection{Perks}

Perks were invented by Chris Taylor after game-designer Brian Fargo suggested levelling up should involve more than simply increasing skill points \autocite{fargo}. Perks are additional rules, or game enhancements, which add an extra layer of sophistication the game's existing rule-set. One can for example add the perk ``animal friend'', which results in wild animals no longer starting combat. This trait requires high intelligence, a high level, and profeciancy in the outdoorsman skill. Another perk one can choose, if the character qualifies to it, is ``Friendly Foe''. This perk helps the player by making friendly characters appear with a green outline when in combat.

\subsection{Levelling up}

During the game, the player will gain experience points which can be used to go up in level. The player can choose to not level up if they wish. At level up, a set amount of skill points can be used to increase the player character's skills. For each third level, the player can choose a new perk.

\subsection{Discovery}

You can switch between two modes, combat mode or discovery mode. This is done with the user interface. In discovery mode, you can speak to characters in the game and learn about where different towns can be found. They will then appear on the world map. You can also access combat mode at any time. You can make friends and enemies in the game world. The game has a system where your reputation and what you do in each town is remembered. Which missions you choose to complete and for whom, will alter the system.

The main quest is to first locate the water chip. Asking characters for information helps the player locate the chip. There are also water traders who can send caravans to your home vault, increasing the time you have to find the water chip.

\subsection{Companions}

You can choose to recruit various companions in the game (4 in total), such as Dogmeat the dog and Ian, a hired gun. In Junktown you can find Tycho, who is aligned with the law. Katja, a scavanger living in the streets, is found in the town known as the Boneyard.

%\subsection{Inventory}

%\subsection{Multiple Quest Solutions}
% https://www.youtube.com/watch?v=vlkksy6casU

\section{Impact on mechanics}

Thanks to the game's systems, a player's decisions have deep consequences on the game's mechanics. In chaos-theory, every little change to a system can cause large changes down the line. This is known as the butterfly effect \autocite{chaos}. Fallout tries to be a chaos system, reacting in deep ways to the player's choices.

\subsection{Weapons}

What skills the character chooses to be good at and improve affect what weapons they should choose to use at any given time and how well the character fares with the selected weapon.

%\subsection{Bartering}

\subsection{Combat}

Fallout has turn-based combat, just like a board game or table-top game. The combat can be activated at any time. Instead of actions relying on dice rolls, characters are given a set amount of action points each turn. Randomness only plays a role when a character decides to attack. The skills and attributes the player has chosen dicatates how well a character fares in combat. Special care was taken to use a random number generator better than the default one found in the ANSI C standard library \autocite{timrandom}.

\subsection{Balance and difficulty}

Having companions makes combat easier. Companion helpers attack and distract enemies. If you choose to kill the sheriff of Junktown, Killian, for the casino boss Gizmo, the recruitable character Tycho will refuse to join your team, limiting the possibilites in the game.

There are good traits and bad traits to choose from at the beginning of the game. Some traits intentionally poor, simply making the game more difficult.

If you are disliked by a faction, you will get attacked by them. The reputation you have therefore directly affects the mechanics. Cutting cornours and solving missions in an overly simple way can make playing more difficult down the line as you miss out on experience points and as the player makes new enemies in the game world.

\subsection{Alliances}

With a character with low intelligence (the I in SPECIAL), the player will not be able to choose all dialog options available. This can make it more difficult to start alliences with factions in the game such as ``The Brotherhood of Steel'' and the various towns found on the world map.

\subsection{Game length and ending}

The game ends if the player runs out of time, dies, or manages to complete the game's main quest for Vault 13. When the game ends, you get a summary of the consequences your actions in the game have had on the characters you have met.

% How do players get choices? How do the choices affect the game?
% Isometric, turn-based, real-time.
% Many choices influence the end of the game, difficulty, society, NPC familiarity, alliances, enemies, game length, player stats/perks, goal: find the water chip, landscapes (where to go)
% what quests to do?
% define rpgs

%As the late mathematician Paul Erdös said, life is suffering \autocite{book:erdos}. Life is hard. Art, however, tends to be overly soft. Children outgrow many of the games they play as they become older as they become better equipped to handle greater difficulties \autocite[99]{book:hiwiller}. By adulthood, one often realizes how unfair and fleeting life can be. Fallout is dark and difficult both to master and to appreciate, it makes you realize how good human beings have it.

%A good research statement serves as a clear and concise summary of the objectives, significance, and methodology of your research. It typically includes the following components:

%Jeg er interessert i hvordan en lager spill der det finnes mange
%interessante historier i spillet som en tar del i, og hvordan disse
%historiene og de moralske valgene en har mulighet til å gjøre
%motiverer spilleren til å delta i spillet.

