%Discussion: Interpretation and analysis of the results in relation to the research question, including a discussion of the implications of the findings, their significance in the context of existing literature, and any limitations of the study.

\section{Discussion}

% systems
% losers getting in the way

French sociologist Roger Caillois defines a video-game as a voluntary rule-based make-believe activity \autocite{caillois_man_2001}. A video-game should always react to the player. Reactivity is the most important aspect of a video-game. Gabe Newell says when a game does not react to something the player does, the player ends up feeling offended. He calls this a narcissistic injury, player choices should always result in a reaction \autocite{gabe_narc}. In Gabe Newell's video-game Half-Life, this is demontrated by, for example, the ability to hit the protagonist's crowbar on the walls. The reaction is a satisfying sound, and a mark left on the wall. In Fallout, you usually have something to do in every level or world you visit. At any time of the game, you can enter combat mode, and use the SPECIAL game-system to make choices, no matter how contrived the game otherwise feels.

Fallout can be considered one of the earliest sandbox games, a genre which rewards creativity and has no predetermined goals, as Fallout does not force any moral decisions on the player \autocite{cain}. The core of the Fallout is figuring out how to spend your time, how to secure a new water chip or shipments of water to your home vault, to survive adversaries in the outside world, and mastering the SPECIAL role-playing system. Discovery is key. Finding what opportunities exist is fundamental to playing the game.
%What are the underlying design principles and player choices in Fallout and what design decisions can be used in other RPGs?

A game like Snakes and Ladders is very unsatisfying to play for adults as each player is completely at the mercy of the dice. Many players do not find such games rewarding as they feel no ownership of the outcome, win or lose \autocite[85]{hiwiller_players_2016}. Fallout solves this problem by presenting the player with the SPECIAL role-playing system. The player gets a set amount of actions which they can use to make various choices. They can decide where to go on the map, what enemies to approach and attack, and which weapon to equip, as well as entering an inventory of player items and making use of various items such as health regenerating steam packs which can be picked up or bartered for in the game world. The player might choose to attempt to flee an enemy encounter. The outcome of the actions feel fair, they give the player a sense of control. As the player combats enemies, the various character attributes chosen by the player at the beginning of the game (strength, perception, endurance, charisma, intelligence, agility, and luck) dictate the outcome of a battles together with the PERKS and traits, the player gets to experiment and try new weapons and get ideas for new characters to try in the game world.

Psychologist Mihaly Csikszentmihalyi proposed the concept of flow to explain the feeling some artists have when they are completely lost in their work \autocite{inbook}. Csikszentmihalyi defines pleasure as the feeling of contentment one feels when ones biological programs and social conditioning have been met, which is felt when you experience something new or have accomplished something. Pleasure improves a person's quality of life and experiencing it consists of the following components:

\begin{enumerate}
\item Tasks with a reasonable chance of completion.
\item Clear goals.
\item Immediate feedback.
\item Deep but effortless involvement that removes from awareness the frustrations and worries of everyday life.
\item Sense of control over our actions.
\item No concern for the self.
\item Alteration of the concept of time, hours can pass in minutes and minutes can look like hours.
\end{enumerate}

A good game should be pleasurable, or as the late Nintendo CEO Satoru Iwata would put it, games should be fun \autocite{iwata}. What that entails varies from person to person. Since Fallout gives the player a sense of control, and it is possible to complete the game, it has clear goals (get the water chip, defeat the mutants) and reacts to our decisions, the game can easily distract from the everyday worries of life. However, a game about nuclear war might not be the perfect distraction in this very day and age.

If you make your own character in Fallout, you will need to retry the game many times to find a character which works in the game world. At the very beginning of the game, you need to make it out of a cave full of rats. This gives the player an opportunity to learn the combat, and test how well the character fares against the easiest enemies in the game. This early mini-climax is an important early payoff for the player which is important in video-games to make players interested while learning the game \autocite[98]{hiwiller_players_2016}.

You can replay the game many times to change how the game ends. The game has an ending, and you get a summary of the consequences your actions in the game have had on the characters you have met. This can motivate the player to play again, and see how the game system changes if they make other choices. Fast turn around times allows for experimentation, I would like to see more this in video-games.

Many older video-games had the ability to save the game state to disk and load your game to continue playing, or playing from an earlier time of playing. Fallout has this feature. The game is notorious for its crash bugs, but saving oftten is a good idea also because one can easily make regretful and unsatisfying decisions, which can be changed by loading the game from an earlier save. Rewinding time might be a mechanic which could be brought back in contemporary video-games.

There are design flaws in Fallout, such as objects blending in with the environment and clunky user interfaces where affordances are unclear and the signifiers are poorly designed. Great designers produce pleasurable experiences \autocite[9]{donald}. Experiences, or sensory analysis, are experienced all the time and determines how people remember interactions. Discoverability is key to good design, it must be possible to construct a conceptual model (a simple explanation of how something works) of a product or system to be able to then use it \autocite[24]{donald}. Without knowing the affordances (the possibliities present to you) and additionally being presented with cumbersome ways of discovering them (poorly designed signifiers), the experience becomes frustrating. There are also unnecessary constraints left in Fallout. Constraints restrict how you interact with a system or product \autocite[67]{donald}. For example, there is no ``take all'' button when you pick up items, making it combersome to loot fallen enemies.

The game world in Fallout has been authored. In authored games, every map or world in the game has been designed by a human-being \autocite[87]{digitale}. This is in contrast to sandbox games such as Minecraft or Dwarf Fortress, where the game world is automatically generated by a computer when you start a new game. Authored games become more polished and have a human touch. Furthermore, the game is completable as it does not have any auto-generated content. It does, however, limit the amount of things one can discover in the game.
